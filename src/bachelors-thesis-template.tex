% Institute of Computer Science thesis template
% authors: Sven Laur, Liina Kamm
% last change Tõnu Tamme 03.05.2019
%--
% Compilation instructions:
% 1. Choose main language on line 55-56 (English or Estonian)
% 2. Compile 1-3 times to get refences right
% pdflatex bachelors-thesis-template
% bibtex bachelors-thesis-template
%--
% Please use references like this:
% <text> <non-breaking-space> <cite/ref-command> <punctuation>
% This is an example~\cite{example}.

\documentclass[12pt]{article}

% A package for setting layout and margins for your thesis 
\usepackage[a4paper]{geometry}

\setlength{\parindent}{0em}
\setlength{\parskip}{1em}
\usepackage{setspace}
\onehalfspace

\usepackage[utf8]{inputenc} %standard encoding since 2018 (can be commented out?)
\usepackage[T1]{fontenc} %Absolutely critical for *hyphenation* of words with non-ASCII letters.

% Typeset text in Times Roman instead of Computer Modern (EC)
\usepackage{times}

% Suggested packages:
\usepackage{microtype}  %towards typographic perfection...
\usepackage{inconsolata} %nicer font for code listings. (Use \ttfamily for lstinline bastype)

\usepackage[english, estonian]{babel} %the thesis is in Estonian

% Change Babel document elements 
\addto\captionsestonian{%
  \renewcommand{\refname}{Viidatud kirjandus}%
  \renewcommand{\appendixname}{Lisad}%
}


% If you have problems with Estonian keywords in the bibliography
%\usepackage{biblatex}
%\usepackage[backend=biber]{biblatex}
%\usepackage[style=alphabetic]{biblatex}
%% plain --> \usepackage[style=numeric]{biblatex}
%% abbrv --> \usepackage[style=numeric,firstinits=true]{biblatex}
%% unsrt --> \usepackage[style=numeric,sorting=none]{biblatex}
%% alpha --> \usepackage[style=alphabetic]{biblatex}
%\DefineBibliographyStrings{estonian}{and={ja}}
%\addbibresource{bachelor-thesis.bib}


% General packages for math in general, theorems and symbols 
% Read ftp://ftp.ams.org/ams/doc/amsmath/short-math-guide.pdf for further information
\usepackage{amsmath} 
\usepackage{amsthm}
\usepackage{amssymb}

% Print a dot instead of colon in table or figure captions
\usepackage[labelsep=period]{caption}

% Packages for building tables and tabulars 
\usepackage{array}
\usepackage{tabu}   % Wide lines in tables
\usepackage{xspace} % Non-eatable spaces in macros

% Including graphical images and setting the figure directory
\usepackage{graphicx}
%\graphicspath{{figures/}}

% Packages for getting clickable links in PDF file
%\usepackage{hyperref}
\usepackage[hidelinks]{hyperref} %hide red (blue,green) boxes around links
\usepackage[all]{hypcap}


% Packages for defining colourful text together with some colours
\usepackage{color}
\usepackage{xcolor} 
%\definecolor{dkgreen}{rgb}{0,0.6,0}
%\definecolor{gray}{rgb}{0.5,0.5,0.5}
\definecolor{mauve}{rgb}{0.58,0,0.82}


% Standard package for drawing algorithms
% Since the thesis in article format we must define \chapter for
% the package algorithm2e (otherwise obscure errors occur) 
\let\chapter\section
\usepackage[ruled, vlined, linesnumbered]{algorithm2e}

% Fix a  set of keywords which you use inside algorithms
\SetKw{True}{true}
\SetKw{False}{false}
\SetKwData{typeInt}{Int}
\SetKwData{typeRat}{Rat}
\SetKwData{Defined}{Defined}
\SetKwFunction{parseStatement}{parseStatement}


% Nice todo notes
\usepackage{todonotes}

% comments and verbatim text (code)
\usepackage{verbatim}

% units
\usepackage{siunitx}
\sisetup{
    output-decimal-marker = {,},
    range-phrase={--},
    range-units=single
}
\DeclareSIUnit \belm {Bm}

% Nice Todo box
\newcommand{\TODO}{\todo[inline]}

%%% BEGIN DOCUMENT
\begin{document}

%===BEGIN TITLE PAGE
\thispagestyle{empty}
\begin{center}


\large
TARTU ÜLIKOOL\\
Arvutiteaduse instituut\\
Informaatika õppekava\\[2mm]


\vspace*{\stretch{5}}
\vspace{25mm}

\Large Kert Tali

\vspace{4mm}

\huge Tartu LPWAN võrkude võrdlus ja kasutusmallid

%\vspace*{\stretch{7}}
\vspace{20mm}

\Large Bakalaureusetöö (9 EAP)

\end{center}

\vspace{2mm}

\begin{flushright}
 {
 \setlength{\extrarowheight}{5pt}
 \begin{tabular}{r l} 
  \sffamily Juhendaja: & \sffamily Alo Peets \\
 \end{tabular}
 }
\end{flushright}

\vspace*{\stretch{3}}
\vspace{10mm}

\vfill
\centerline{Tartu 2020}

%===END TITLE PAGE

% If the thesis is printed on both sides of the page then 
% the second page must be must be empty. Comment this out
% if you print only to one side of the page comment this out
%\newpage
%\thispagestyle{empty}    
%\phantom{Text to fill the page}
% END OF EXTRA PAGE WITHOUT NUMBER


%===COMPULSORY INFO PAGE
\newpage

%=== Info in English
\newcommand\EngInfo{{%
\selectlanguage{english}
\noindent\textbf{\large Comparison and use cases of LPWAN networks in Tartu}
\vspace*{1ex}

\noindent\textbf{Abstract:}
\TODO{Valmib lõpus}
\noindent

\vspace*{1ex}

\noindent\textbf{Keywords:}\\
IoT, LPWAN, LoRaWAN, Sigfox, NB-IoT, service comparison, use cases

\vspace*{1ex}

\noindent\textbf{CERCS:} T180 Telecommunication engineering; P170 Computer science, numerical analysis, systems, control

\vspace*{1ex}
}}%\newcommand\EngInfo


%=== Info in Estonian
\newcommand\EstInfo{{%
\selectlanguage{estonian}
\noindent\textbf{\large Tartu LPWAN võrkude võrdlus ja kasutusmallid}
\vspace*{1ex}

\noindent\textbf{Lühikokkuvõte:} 

%\noindent ...
\TODO{Valmib lõpus}
%\TODO{One or two sentences providing a basic introduction to the field, comprehensible to a scientist in
%any discipline.}
%\TODO{Two to three sentences of
%more detailed background, comprehensible to scientists in related disciplines.}
%\TODO{One sentence clearly stating the general problem being addressed by this particular
%study.}
%\TODO{One sentence summarising the main result (with the words ``here we show´´ or their equivalent).}
%\TODO{Two or three sentences explaining what
%the main result reveals in direct
%comparison to what was thought to be the case previously, or how the main result adds to previous knowledge.}
%\TODO{One or two sentences to put the results into a more general context.}
%\TODO{Two or three sentences to provide a
%broader perspective, readily
%comprehensible to a scientist in any
%discipline, may be included in the first paragraph
%if the editor considers that the accessibility of
%the paper is significantly enhanced by their inclusion.}

\vspace*{1ex}

\noindent\textbf{Võtmesõnad:}\\
IoT, LPWAN, LoRaWAN, Sigfox, NB-IoT, teenusvõrkude võrdlus, kasutusmallid

\vspace*{1ex}

\noindent\textbf{CERCS:} T180 Telekommunikatsioonitehnoloogia; P170 Arvutiteadus, arvutusmeetodid, süsteemid, juhtimine (automaatjuhtimisteooria)

\vspace*{1ex}
}}%\newcommand\EstInfo


%=== Determine the order of languages on Info page
{\EngInfo}{\EstInfo}


\newpage
\setlength{\parskip}{0em}
  \tableofcontents
\setlength{\parskip}{1em}

% Remember to remove this from the final thesis version
%\newpage
%\listoftodos[Unsolved issues]
% END OF TODO PAGE 


\newpage
\section{Sissejuhatus}

%\TODO{What is it in simple terms (title)?}
%\TODO{Why should anyone care?}
%\TODO{What was my contribution?}
%\TODO{What you are doing in each section (a sentence or two per section)}
Nutistu üha suurem tähtsus äri- ja mugavusrakendustes põhjustavad ühest küljest võidujooksu uute lahenduste leiutamisele, teisest küljest võitluse tehnoloogiliste ja füüsiliste piirangutega.
Üks sellistest pudelikaeladest hõlmab seadmetevahelist suhtlust (M2M), mille puhul tuleb valida olemasolevate sideprotokollide seast oma rakendusele sobivaim.
Traadita andmeside tehnoloogiate seas on populaarsust kogumas uus klass — LPWAN (Low Power Wide Area Network), mis on mõeldud just rakendustele, mis eeldavad nii madalat voolutarvet kui laia leviala.
Populaarseimad LPWAN tehnoloogiad on Sigfox, LoRa, NB-IoT, LTE-M, Ingenu ja Weightless.

Kuigi tehnoloogiad on orienteeritud ühele niššile, erinevad need oma tehnilise spetsifikatsiooni poolest märkimisväärselt.
Uurimistöö eesmärk on hinnata Sigfox, LoRa ja NB-IoT tehnoloogiate sobivust erinevatele IoT kasutusjuhtudele 2020. aastal Tartu linna ja lähiümbruse näitel.
Uuringu meetod on riistvara katsetamine, mille käigus katsetatakse kättesaadavaid teenuseid ning leitakse tehnoloogiate piirid töökindluse ja levi osas.
Seetõttu on uurimus väärtuslik ülevaade konkureeruvate tehnoloogiate reklaamitava jõudluse kõrvutamisest tegelikuga.
Kasutusjuhtude omistamisel võetakse arvesse mõõtmistulemusi, olemasoleva infrastruktuuri sobivust ning tehnoloogia kasutuselevõtule kulunud aega ja raha.

\TODO{Ülevaade tehtud töödest \ldots}
\TODO{Minu töö uudsus ja vajalikkus \ldots}
\TODO{Töö jaotub peatükkideks \ldots}

\newpage
\section{Taust}

Gaddami ja Rai väitel on LPWANi võrgud loodud, pidades silmas neid võrku ühendatavaid seadmeid ja nende kasutusmalle, millele ei sobi traditsioonilised traadita andmeside tehnoloogiad nagu 802.11 (WiFi) perekond, Bluetooth või senised 3GPP mobiilside standardid [1].
Tuuakse välja, et uute sidelahenduste välja töötamine on tingitud eelkäijate võtmeomadustest, keskendudes eelkõige  suuremale andmeedastuskiirusele ja minimaalsele latentsile.

Gubbi jt visioonis asjade interneti (Internet Of Things ehk IoT) tuleviku kohta defineeritakse seda kui võrgustikku omavahel ühenduses olevaid seadmeid — sensorid, mis koguvad oma keskkonna kohta infot ning aktuaatorid interaktsiooniks teise seadmega, kusjuures suhtlus toimub üle olemasolevate Internetistandardite [5].
Augustin jt. eristavad Internetti ja asjade internetti väites, et IoT seadmetel on palju vähem ressursse, kui Interneti seadmetel [6].
Ressurssidena tuuakse välja eelkõige jõudlus ja mälu, kuid eraldiseisvate üksuste puhul ka vooluallikas, milleks on sageli akud või näiteks päikesepatareid [7].

Chang jt esitavad pilvepõhise IoT süsteemide disainis viis väljakutset: kitsas ribalaius, kõrge latents, side ebastabiilsus, ressursside piiratus ja turvalisus [2].
Kolm esimest on otseselt seotud kasutatava võrgu füüsilise kihiga, ehk disainiotsused andmevahetuse korraldamisel sõltuvad tugevalt valitud sidekanali omadustest.
Nilsson ja Svensson, kes keskenduvad oma artiklis RF transsiiverite voolutarbimise optimeerimisele, omistavad eelnevalt mainitud probleemid just IoT seadmete vooluallikatele, mis ei suuda ära toita võimsat ja keerukat raadiokommunikatsiooni [3].
Sama väidet toetavad Chen jt, kes tõdevad, et värkvõrgu minimaalne riistvara on kujundanud sellise olustiku, kus väga erinevate parameetrite ja piirangutega IoT tüüplahendustele disainitakse uusi, võimalikult optimeeritud sidetehnoloogiaid [4].
Näiteks tuuakse energiasäästlikud LPWAN tehnoloogiad, mis on suunatud IoT lahendustele, mis koosnevad suurest hulgast akutoitega seadmetest, laialijaotatuna suurele maa-alale.

\subsection{Kasutusmallid}

Vajadus autonoomsete mõõtmisseadmete järele on paljudel eluvaldkondadel. Enim paistavad silma LPWANi jaoks soodsa tegevuspiirkonnaga põllumajandusrakendused, mis tingivad andmete kogumist suurel pindalal ning suurelt arvult sensoritelt, milleni pole otstarbekas vedada kaableid.
Jawad jt. [8] toovad välja põhilised rakendusalad, mis aitavad planeerida täppispõllumajandust ning on saavutatavad suurel mastaabil vaid traadita seadmetega: pinnase ja õhu niiskuse, temperatuuri ja mulla viljakuse mõõtmine.
Lisaks vajaliku info üleslink saatmisele tuleb teatud põllumajanduslikele rakendustele kasuks, kui seadmed on võimelised ka allalink suhtluseks ehk kuulama instruktsioone.
Seda omadust on uuritud põllumajanduses näiteks tarkade niisutussüsteemide või kasvuhoonetes tuulutussüsteemide rakendamisel [9].

Samasugustes mastaapsetes tingimustes kasutatakse IoT sensoreid keskkonnanäitajate kogumiseks.
Selleks paigaldatakse looduskeskkonda — näiteks jõgedesse, veekogudesse, metsadesse, lagendikele või linnade äärealadele sensoreid, mis raporteerivad jõe voolu [10], vee kvaliteeti [11], vee taset [12], õhukvaliteeti [13] ja muud.
Samuti on sellistel sensorvõrkudel potentsiaal kaardistada ja ennustada looduskatastroofe, nagu  maastikupõlengud~\cite{kang}.

Sarker jt. sõnastavad Smart-city kui ühe IoTga tihedalt seotud kontseptsiooni — rakendada suurel hulgal erinevaid mõõtmissüsteeme linnaplaneerimise hõlbustamiseks [14].
Viidatakse sellele, et see pole küll uus idee, kuid LPWAN tehnoloogiate tulekuga on tekkimas üha enam uusi ettepanekuid targa linna süsteemidele.
Staatilisi sensoreid saab linnapildis rakendada näiteks prügikonteinerite täituvuse, liiklustiheduse, helireostuse ja parklate täituvuse raporteerimiseks~\cite{zanella}.
GPRS baasil kaugloetavad elektriarvestid on Eesti kodudes kohustuslikud alates 2017. aastast~\cite{laurit} ning 2021. aastaks peavad välja vahetatud olema ka küttegaasi arvestid~\cite{gaas}.
Sealjuures on raskendatud leviga kortermajade või keldrite arvestid enamasti ühendatud kontsentraatoritega, mis kannavad andmed traadita andmesideks sobivamasse keskkonda.

Staatiliste seadmete kõrval on olulisel kohal ka mobiilsed rakendused, mis eelnevale sarnaselt monitoorivad ennast kandvat sõidukit või looma, või on hoopis positsioneerimise otstarbega.
Väikseid akutoitega seadmeid on kerge kinnitada loomale või inimesele, et saada automaatseid tervisenäitajaid loomapidamises~\cite{germani, liliu} või tervishoius~\cite{olatinwo, petajajarvi}.
Samuti on mis tahes objektide koordinaatide raporteerimine GNSS mooduli abil potentsiaalne kasutusala LPWAN põhiseks positsioneerimiseks, mille potentsiaali on demonstreeritud jalgrattaringluse kontekstis~\cite{kimpark} või dementse inimese leidmiseks~\cite{hadwen}.
Liikuvate seadmete ohud väljenduvad erijuhtudena, kui seadmed liiguvad levialast välja või satuvad läbistamatute takistuste taha ja andmeside katkeb, mistõttu kasutatav tehnoloogia peab olema piisavalt robustne ja laia katvusega.

Kusuda ja Iwata esitavad idee liigutada hoopis tugijaamu paiksete lõppseadmete suhtes~\cite{kusuda}.
Süsteemi võimalike otstarvetena näevad autorid: prügiauto ja prügikastide vahelist suhtlust, et optimeerida korjeteekonda, drooniga pestitsiidide kasutuse tuvastamine ülisuurelt alalt ja kullerteenuse-majapidamise vaheline eelhoiatus, kui auto jõuab naabruskonda.
Sellise süsteemi eelis on levi kindlustamine ettenähtud kohtades, samaaegselt lõppseadmetele pikka eluiga garanteerides.

\subsection{Side optimeerimise meetodid}

Lai leviulatus korraga akusäästlikusega on vasturääkivus, sest eeldab justkui võimsa ja kuluka raadiosignaali genereerimist.
Selleks, et tehnoloogia täidaks LPWANi tingimusi, on eelkõige oluline, et sideprotokoll võimaldaks lõppseadmele pikka eluiga.
Sealjuures on oluline küll distants, kuid muude tegurite osas võib teha olulisi kärpimisi.
Selles alapeatükis käsitletakse tehnilisi iseärasusi ehk võtteid, mida kasutatakse LPWAN tehnoloogiate disainis.
Järjestuse aluseks on analoogiline loetelu Raza jt 2017. aasta ülevaatest LPWAN tehnoloogiatest~\cite{raza}.

Signaali levi- ja läbistavusomaduse põhiline tegur on kasutatav sagedus — madalamad sagedused on vastupanelikumad takistuste ja hajumise osas, mistõttu on mitmed LPWAN tehnoloogiad loodud töötama alla \SI{1}{\giga\hertz} sagedustel, mitte enamlevinud \SI{2,4}{\giga\hertz} spektris, mida kasutavad WiFi ja Bluetooth ning ka mõned IoT võrgud nagu Zigbee~\cite{bardyn}.
Teine viis levivõimekuse suurendamiseks hõlmab signaali kodeerimise- ja modulatsioonitehnikat, ehk OSI mudeli mõistes võrgu esimest (e. füüsilist) kihti. Selleks, et suurendada seadme võimalust demoduleerida ka väga nõrka signaali, luuakse sageli spetsialiseeritud modulatsiooniskeemid, mis on piisavalt vastupanelikud raadiohäiretele ja signaali hajumisele~\cite{reynders}.
Selline üle raadioside kantav bitijada peab piltlikult olema selge ja robustne, omades väikest andmeedastuskiirust.

Voolutarbes saavutatakse võit enamasti tänu lõppseadme suurele passiivses olekus veedetud aja osakaalule, kui andmeid ei saadeta ega võeta vastu.
Anastasi jt~\cite{anastasi} uurimuses IoT silmusvõrkude voolutarbest tõdetakse, et raadio on eranditult suurim tarbija lõppseadmes, mille ärkvelolekut tuleb võimalikult palju piirata.
Kui silmusvõrkudes kulub suur osa lõppseadme ressursist naaberseadme retransleerimisele, siis LPWAN võrgud töötavad üldiselt vaid tähtvõrguna,  milles lõppseade suhtleb ainult tugijaamadega~\cite{centenaro}.
Selle tõttu on LPWANidel palju vabam süvaune planeerimine, sest iga seade hoolitseb vaid enda andmevahetuse eest. Voolutugevus, kui seade on süvaunes, peaks olema umbes \SI{1}{\micro\ampere}~\cite{goursaud}.

Levinud voolusäästmise meetod on ALOHA meediumipöördus, mille puhul ei kasutata meetmeid, et vältida naaberseadmega konfliktimist, vaid minnakse eetrisse esimesel ettejuhtuval ajal~\cite{raza}.
Ühelt poolt on see käitumine odav, kuid suureneb pakettide põrkumise tõenäosust ja seetõttu on tarvis lisameetmeid side usaldusväärsuse tagamiseks, nagu pakettide kordamine suvalistel kanaliltel Sigfoxi näitel~\cite{raza}.

Ultrakõrgsagedusalas sobilikud litsentseerimata sagedusvahemikud on enamasti ISM (Industrial-Scientific-Medical) ribad, mis on reguleeritud seadmete eetrihõive ja kiirgusvõimsuse osas — Euroopas kehtib näiteks Sigfoxi ja LoRa poolt kasutatavale \SI{868}{\mega\hertz} sagedusele seadusega sätestatult 1\% maksimaalne päevane täitetegur ja \SI{25}{\milli\watt} kiirgusvõimet seadme kohta~\cite{etsi}.
See tähendab, et transsiiver võib olla eetris kokku kuni ühe sajandiku ööpäevast, sealhulgas mitte ületada sätestatud võimsust.
LPWAN kasutusalad suurest eetrihõivest ei võida, vastupidi on piirangud kooskõlas madala voolutarbe nõudega.



\newpage
\section{Olemasolevad tehnoloogiad}

Käesolevas peatükis antakse üksikasjalik ülevaade hetkel aktuaalsetest LPWANi perekonda kuuluvatest tehnoloogiatest, nende eelistest ja puudustest.

\subsection{Sigfox}

Sigfox on 2010. aastal Prantsusmaa start-upi poolt loodud tehnoloogia, mis loojate visiooni kohaselt tagab ühenduse miljarditele IoT seadmetele, luues minimaalne telekommunikatsioonistandardi pisikeste sõnumite edastamiseks [15].
Ettevõte pakub tehnoloogiat ja pilvekeskkonda komplektse teenusena [16].

Sigfox töötab Euroopas \SIrange{868.0}{868.2}{\mega\hertz} mittelitsentseeritud sagedusalas ning kasutab ülikitsaribalist (Ultra-Narrow-Band ehk UNB) side, milles iga üleslink ülekanne kasutab vaid 100-hertsi laiust riba [17].
Kalfusi ja Hegri hinnangul võimaldab UNB Sigfoxil teiste tehnoloogiate seas esile tõusta mürataluvuse ja saavutatava distantsi osas [18].
Põhjuseks on, et kitsale vahemikule satub harvemini raadiohäireid, mis omakorda alandab üldist ülekande nurjumise tõenäosust ja laseb edukalt vastu võtta kuni \SI{-142}{\deci\belm} hääbunud signaali (RSSI).
\SI{100}{\hertz} laiused sõnumid jaotatuna \SI{0,2}{\mega\hertz} ribale tähendab, et pääsupunkt kuulab samaaegselt ligi 2000 teoreetilisel vahemikul, pakkudes küllaldaselt ribalaiust samaaegseteks ülekanneteks ka kasutataval ALOHA mehhanismil.

Teenus lubab maksimaalselt 12 baidi pikkuseid sõnumeid, seadme kohta kuni 140 üleslink ja nelja kaheksabaidist allalink paketti päevas, kusjuures allalink suhtlus on võimalik ainult vastuvõtuaknas, mis avaneb lühidalt peale üleslink sõnumi saatmist.
Piirangutega vähendatakse seadmete eetriaega, sõnumite põrkumise tõenäosust ning kokkuvõttes tagatakse võrgu töökindlus ka suurema liikluse puhul.
Ühe 12 baidise paketi saatmine võtab aega 2,08 sekundit, mida korratakse kolmel erineval sagedusel, mis tähendab halvimal juhul kuni 6,24 sekundilist latentsi.

Sigfoxi tehnoloogia kasutamine on võimalik ainult läbi ettenähtud ökosüsteemi.
Kasutaja peab registreerima end, oma seadmed ning tasuma seadmepõhist ühendustasu.
Leviala eest vastutavad Sigfoxi partnerid erinevates riikides, kes haldavad tugijaamu — Eestis Connected Baltics~\cite{sfCoverage}.
Rakendusliidesega on võimalik korraldada suhtlust füüsiliste seadmetega, suunates kuuldud paketid oma rakendustesse ning saata lõppseadmetele sõnumeid tagasi.

\subsection{LoRaWAN}

LoRa (tuletatud \textit{Long Range}) on nimetus füüsilisele kihile, mille on välja arendanud USA ettevõte Semtech ~\cite{loraIntro}.


\subsection{NB-IoT}

\subsection{Muud tehnoloogiad}

\newpage
\section{Keskkonna ülesseadmine}

\subsection{Arendusplaat FiPy}

\subsection{LoRaWAN tugijaam}

\subsection{Kasutusmallide testimine}

\subsubsection{Maapiirkond}

\subsubsection{Linnapiirkond}

\subsubsection{Liikuv tugijaam}

\newpage
\section{Kokkuvõte}

\newpage

% BibTeX bibliography
\bibliographystyle{unsrt} %plain=[1], alpha=[BGZ09]
\bibliography{../src/bachelor-thesis}

\addcontentsline{toc}{section}{\refname}


% Use Biblatex if you have problems with Estonian keywords
%\printbibliography %biblatex


\newpage
\appendix
%\section*{\appendixname}
\section*{Lisad}
  \addcontentsline{toc}{section}{Lisad}


\section*{I. Testskriptid}
\addcontentsline{toc}{subsection}{I. Lõppseadme testskriptid}

\newpage

%=== Licence in Estonian
\newcommand\EstLicence{{%
\section*{II. Litsents}

\addcontentsline{toc}{subsection}{II. Litsents}

\subsection*{Lihtlitsents lõputöö reprodutseerimiseks ja üldsusele kättesaadavaks tegemiseks}

Mina, \textbf{Kert Tali},

\begin{enumerate}
\item
annan Tartu Ülikoolile tasuta loa (lihtlitsentsi) minu loodud teose
\par
\textbf{Tartu LPWAN võrkude võrdlus ja kasutusmallid},
\par
mille juhendaja on Alo Peets,
\par
reprodutseerimiseks eesmärgiga seda säilitada, sealhulgas lisada digitaalarhiivi DSpace kuni autoriõiguse kehtivuse lõppemiseni.
\par
\item
Annan Tartu Ülikoolile loa teha punktis 1 nimetatud teos üldsusele kättesaadavaks Tartu Ülikooli veebikeskkonna, sealhulgas digitaalarhiivi DSpace kaudu Creative Commonsi litsentsiga CC BY NC ND 3.0, mis lubab autorile viidates teost reprodutseerida, levitada ja üldsusele suunata ning keelab luua tuletatud teost ja kasutada teost ärieesmärgil, kuni autoriõiguse kehtivuse lõppemiseni.
\item
Olen teadlik, et punktides 1 ja 2 nimetatud õigused jäävad alles ka autorile.
\item
Kinnitan, et lihtlitsentsi andmisega ei riku ma teiste isikute intellektuaalomandi ega isikuandmete kaitse õigusaktidest tulenevaid õigusi. 
\end{enumerate}

\noindent
Kert Tali\\ %author's name
\textbf{\textsl{pp.kk.aaaa}}
}}%\newcommand\EstLicence


\EstLicence


\end{document}

